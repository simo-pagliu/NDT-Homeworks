\documentclass[11pt,a4paper,twoside]{article}

% Input encoding and basic packages
\usepackage[utf8]{inputenc}
\usepackage{amsmath, amssymb}
\usepackage{graphicx}
\usepackage{tcolorbox}
\usepackage{xcolor}
\usepackage{geometry}
\usepackage{wrapfig}
\usepackage{titlesec}
\usepackage{fancyhdr}
\usepackage{caption}
\usepackage{float}
\usepackage{hyperref}
\usepackage{subcaption}
\usepackage{fontspec}
\usepackage{nomencl} 
\usepackage{multicol}
\usepackage{etoolbox}
\usepackage{sectsty} % Allows custom section styles
\usepackage{helvet}  % Default sans-serif font
\usepackage{times}   % Example serif font for body text

% Define custom fonts
\setmainfont{Poppins} % Main font for body text
\newfontfamily\titlesfont{Anonymous Pro} % Font for titles

% Relax Latex rules for text layout
\raggedbottom
\sloppy
\hbadness=10000

% Define custom colors
\definecolor{cherenkovblue}{RGB}{55, 139, 230}

% Adjust default font sizes (Explicitly set sizes)
\renewcommand{\normalsize}{\fontsize{9}{11}\selectfont} % Smaller body text: 9pt font with 11pt line spacing
\renewcommand{\small}{\fontsize{8}{10}\selectfont}
\renewcommand{\footnotesize}{\fontsize{7}{9}\selectfont}

% Make titles all caps
\makeatletter
\let\oldsection\section
\renewcommand{\section}{\@startsection {section}{1}{\z@}%
  {-3.5ex \@plus -1ex \@minus -.2ex}%
  {2.3ex \@plus.2ex}%
  {\titlesfont\fontsize{14}{16}\bfseries\color{cherenkovblue}\MakeUppercase}}
\makeatother

% Custom subsection font
\subsectionfont{\titlesfont\fontsize{12}{14}\bfseries\color{cherenkovblue}} % Subsection: 12pt font, bold, blue

% Customize spacing
\setlength{\parskip}{6pt} % Add spacing between paragraphs
\setlength{\parindent}{0pt} % Remove paragraph indentation
\setlength{\itemsep}{4pt} % Add spacing between list items

% Customize itemize bullets
\renewcommand\labelitemi{--} % Use dashes instead of bullets

% Define fancy header and footer
\fancypagestyle{plain}{
    \fancyhf{} % Clear all header and footer fields
} % Default plain style for title and index pages

\fancypagestyle{post-index}{
    \fancyhf{} % Clear all header and footer fields
    \fancyhead[LE]{\textit{S.Pagliuca, T.Pirola, L.Raffuzzi, R.Ronchi, D.Shabi}}
    \fancyhead[RO]{\textit{Nuclear Design and Technologies}}
    \fancyfoot[LE,RO]{\thepage}
    \renewcommand{\headrulewidth}{0.4pt}
    \renewcommand{\footrulewidth}{0.4pt}
    \setlength{\headheight}{14pt}
}

\begin{document}
%%%%%%%%%%%%%%%%%%%%%%%%%%%%%%%%%%%%%%%%%%%%%%%%%%%%%%%%%%%%%%%%%%%%%%%%%%%%%%%%%%%%%%%%%%%%
% First page
% Includes title, author, abstract, keywords, nomenclature
\thispagestyle{plain}

% Title with Titles Font (Anonymous Pro)
{\titlesfont\fontsize{18}{28}\textbf{\color{cherenkovblue}{FUEL PIN \\ PRELIMINARY DESIGN}}}\\
{\titlesfont\fontsize{10}{12}\color{cherenkovblue} Nuclear Engineering - Politecnico di Milano}\\


\vspace{-10pt}

% Author information with Main Font (Poppins)
{\normalsize\textbf{Simone Pagliuca, Tommaso Pirola, Lisa Raffuzzi, Riccardo Ronchi, Darien Shabi}} \\
{\footnotesize\textit{simone1.pagliuca@mail.polimi.it}} \\

\vspace{10pt}

% Course and Academic Year
{\footnotesize\textbf{Course:} Nuclear Design and Technologies}\\
{\footnotesize\textbf{Academic year:} 2024/2025}

% Horizontal rule
\vspace{8pt}
\centerline{\rule{1.0\textwidth}{0.4pt}}

\vspace{10pt}

% Abstract with Titles Font for Heading, Main Font for Body
{\fontsize{8}{8}\textbf{\color{cherenkovblue} ABSTRACT:}} 
{\normalsize
    lorem ipsum dolor sit amet, consectetur adipiscing elit. Donec auctor, nunc nec ultricies ultricies, nunc nunc.
}

\vspace{10pt}

% Key-words box with Titles Font
\begin{tcolorbox}[arc=0pt, boxrule=0pt, colback=cherenkovblue!60, width=\textwidth, colupper=white]
    {\titlesfont\fontsize{10}{10}\textbf{Key-words:}} Key, Words, Here
\end{tcolorbox}

\vspace{10pt}

% Nomenclature Definitions
\makenomenclature
\renewcommand\nomgroup[1]{%
\item[\bfseries
\ifstrequal{#1}{A}{A Quantities}{%
\ifstrequal{#1}{B}{B Quantities}{}}%
]}

% Two-column layout for the nomenclature
\renewcommand{\nompreamble}{\begin{multicols}{2}}
\renewcommand{\nompostamble}{\end{multicols}}

% Define Nomenclature Entries
\nomenclature[A, 01]{$x$}{X quantity}
\nomenclature[B]{$y$}{Y quantity}

% Render Nomenclature Without Section Break
\vspace{-5pt} % Adjust spacing to fit table
\printnomenclature

\newpage
%%%%%%%%%%%%%%%%%%%%%%%%%%%%%%%%%%%%%%%%%%%%%%%%%%%%%%%%%%%%%%%%%%%%%%%%%%%%%%%%%%%%%%%%%%%%

%%%%%%%%%%%%%%%%%%%%%%%%%%%%%%%%%%%%%%%%%%%%%%%%%%%%%%%%%%%%%%%%%%%%%%%%%%%%%%%%%%%%%%%%%%%%
% Table of contents
% Switches to the post-index page style after TOC
\tableofcontents
\newpage
\pagestyle{post-index}
%%%%%%%%%%%%%%%%%%%%%%%%%%%%%%%%%%%%%%%%%%%%%%%%%%%%%%%%%%%%%%%%%%%%%%%%%%%%%%%%%%%%%%%%%%%%

%%%%%%%%%%%%%%%%%%%%%%%%%%%%%%%%%%%%%%%%%%%%%%%%%%%%%%%%%%%%%%%%%%%%%%%%%%%%%%%%%%%%%%%%%%%%
% Content
\section{Introduction}
The fuel pin design process involves:
\begin{itemize}
    \item Determining the cladding thickness, the fuel-cladding gap size, and the plenum height.
    \item Verifying the design against limits for fuel melting, cladding temperature, yielding, and thermal creep.
    \item Identifying critical aspects if the irradiation time is doubled.
\end{itemize}
Most design specifications were provided. Missing data were sourced from relevant literature or handouts.

\section{Assumptions and Methodology}
\textbf{Material and Thermal Assumptions:}
\begin{itemize}
    \item Material properties were treated as temperature-dependent, modeled using empirical correlations.
    \item Axial profiles for power and neutron flux were assumed constant.
    \item Initial helium pressure and temperature in the fuel-cladding gap were as specified.
\end{itemize}
\textbf{Key Approximations:}
\begin{itemize}
    \item The fuel-cladding gap was treated as uniform, neglecting axial deformation.
    \item Fission rate calculations assumed uniform flux and macroscopic cross sections derived from the JANIS database.
    \item Plutonium redistribution and void swelling were modeled using analytical correlations.
\end{itemize}

\section{Design Results and Verification}
\subsection{Preliminary Sizing}
The genetic algorithm was used to optimize the cladding thickness and plenum height while maintaining performance and safety:
\begin{itemize}
    \item \textbf{Cladding Thickness:} 80–120 $\mu$m.
    \item \textbf{Plenum Height:} 80–100 cm.
\end{itemize}

\subsection{Verification}
\textbf{Thermal Performance:}
\begin{itemize}
    \item The margin to melting of the fuel was validated under steady-state conditions.
    \item Axial and radial temperature profiles confirmed acceptable temperature gradients in the cladding and fuel.
\end{itemize}
\textbf{Mechanical Analysis:}
\begin{itemize}
    \item Stresses in the cladding were calculated using Mariotte and Lamé criteria.
    \item The time to rupture due to thermal creep was evaluated using the Larson-Miller parameter.
\end{itemize}
\textbf{Key Findings:}
\begin{itemize}
    \item The fuel-cladding gap remained within the design range throughout operation.
    \item Cladding stresses remained below yielding limits.
    \item The design provided sufficient safety margins for thermal creep and mechanical integrity.
\end{itemize}

\section{Critical Issues for Extended Operation}
If the irradiation time is doubled:
\begin{itemize}
    \item Increased fission gas release (FGR) will elevate internal pressure and reduce thermal conductivity in the gap.
    \item Cladding thickness constraints must be re-evaluated to prevent long-term creep failure.
    \item The redistribution of Plutonium could lead to localized power peaking, necessitating further analysis.
\end{itemize}
Conservative limits on cladding thickness and plenum height were implemented to address these issues.

\section{Conclusions}
The finalized design meets all specified requirements, ensuring safety and reliability for up to a four-year fuel cycle. Conservative assumptions and thorough validation steps provided additional safety margins. The design demonstrates robustness under normal and extended operational conditions.

\section*{Appendix}
All supporting code is available in the following repository:
\href{https://github.com/simo-pagliu/NDT-Homeworks}{\textcolor{blue}{NDT-Homeworks Repository}}.

%%%%%%%%%%%%%%%%%%%%%%%%%%%%%%%%%%%%%%%%%%%%%%%%%%%%%%%%%%%%%%%%%%%%%%%%%%%%%%%%%%%%%%%%%%%%
\end{document}
