
\section{Verification of Models}
Each function used in the analysis was tested for expected behavior.

Below are the key verification steps.

\subsection{Thermal-Hydraulics Analysis}
Preliminary checks were conducted on coolant properties.

Density, viscosity, and thermal conductivity were validated at the coolant inlet temperature.

The heat transfer coefficient between the coolant and cladding was computed using Nusselt number correlations.

Figure~\ref{fig:thermal_hydraulics} shows the axial power profile and the computed heat transfer coefficient.

\begin{figure}[H]
\centering
\includegraphics[width=0.8\textwidth]{placeholder.png}
\caption{Thermal-hydraulics analysis results.}
\label{fig:thermal_hydraulics}
\end{figure}

\subsection{Temperature Profiles}
Radial and axial temperature profiles were computed for the cold geometry to check if the shape of the profile made sense and if the values were reasonable.

The results provided an initial validation of the temperature distribution behavior before further analysis.

Figure~\ref{fig:temperature_profile} illustrates the temperature distributions.

\begin{figure}[H]
\centering
\includegraphics[width=0.8\textwidth]{placeholder.png}
\caption{Radial and axial temperature profiles for cold geometry.}
\label{fig:temperature_profile}
\end{figure}

\subsection{Fission Gas Behavior}
Rate theory equations were solved to estimate fission gas production and release over the fuel cycle.

The rate equation for the gas remaining in the grains was solved under steady state conditions, given that we are only interested in sizing the plenum for 1-year operation.

The fission rate was calculated as the product of the macroscopic cross section and the average flux.

The macroscopic cross section was obtained from data in the JANIS database, and the average flux was estimated by assuming the power and flux profiles were equivalent.

The concentration profiles are shown in Figure~\ref{fig:fission_gas}.

\begin{figure}[H]
\centering
\includegraphics[width=0.8\textwidth]{placeholder.png}
\caption{Fission gas concentration profiles.}
\label{fig:fission_gas}
\end{figure}
