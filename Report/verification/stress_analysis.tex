\section{Mechanical Analysis - Stress Assessment}

\subsection{Introduction}
This section focuses on the mechanical analysis of the cladding, evaluating stresses to ensure compliance with safety limits during operation. The analysis includes stress computations based on the Mariotte and Lamé criteria, an assessment of plastic strain potential, and an evaluation of rupture time due to thermal creep.

\subsection{Methodology}
The stress distribution in the cladding was calculated using elasticity equations for cylindrical geometries, under the assumptions of orthocylindricity and axial symmetry.

\subsubsection{Mariotte Criterion}
\begin{itemize}
    \item[$\hookrightarrow$] Evaluates the hoop stress at the mid-wall of the cladding.
    \item[$\hookrightarrow$] Used to check for yielding. Preliminary calculations indicated that the Mariotte stress is slightly higher than the Lamé stress, although the latter imposes stricter criteria.
\end{itemize}

\subsubsection{Lamé Criterion}
\begin{itemize}
    \item[$\hookrightarrow$] Considers radial, hoop, and axial stresses.
    \item[$\hookrightarrow$] The radial stress is computed using the pipe equation with a superposition solution:
    \begin{equation}
        \frac{d}{dr} \left( r^3 \cdot \frac{d\sigma_r(r)}{dr} \right) + \frac{\alpha E}{1 - \nu} \cdot \frac{dT(r)}{dr} = 0
    \end{equation}
    Using equilibrium relations, the hoop stress is determined as:
    \[
    \sigma_\theta = \sigma_r + \frac{d\sigma_r}{dr}
    \]
    \item[$\hookrightarrow$] The Tresca criterion is applied for verification, ensuring compliance with yield and ultimate strength limits. The Lamé criterion was found to impose limits approximately 33\% stricter than the Mariotte criterion.
\end{itemize}

\subsubsection{Plastic Strain Check}
\begin{itemize}
    \item[$\hookrightarrow$] Plastic strain is flagged if either criterion predicts stresses exceeding the yield strength.
    \item[$\hookrightarrow$] No significant plastic strain was observed when both criteria were satisfied.
\end{itemize}

\subsubsection{Thermal Creep (Time to Rupture)}
\begin{itemize}
    \item[$\hookrightarrow$] The rupture time due to thermal creep was evaluated using the Larson-Miller Parameter (LMP), based on operating stresses and temperatures.
    \item[$\hookrightarrow$] Even under conservative assumptions, the calculated time to rupture showed sufficient margins, indicating minimal risk of creep-related failure.
\end{itemize}

\subsection{Results and Conclusion}

\subsubsection{Stresses from Mariotte and Lamé}
\begin{itemize}
    \item[$\hookrightarrow$] Both criteria confirmed that the stresses remain below critical limits.
    \item[$\hookrightarrow$] While the Mariotte stress tends to be marginally higher, it remains within safe bounds. Lamé's stricter limits provide an additional safety margin.
\end{itemize}

\subsubsection{Plastic Strain}
\begin{itemize}
    \item[$\hookrightarrow$] No significant plastic strain was observed, confirming that the cladding operates within the elastic regime under specified conditions.
\end{itemize}

\subsubsection{Time to Rupture}
\begin{itemize}
    \item[$\hookrightarrow$] The LMP-based calculation indicated that the cladding's operational life significantly exceeds the irradiation period, even under conservative conditions.
\end{itemize}

These analyses validate the mechanical reliability of the cladding in the initial design. The results demonstrate that the proposed design is safe and robust, with sufficient margins for long-term operation. Further refinement of the design will allow for additional verification and optimization.
