\subsection{Results}

The design and verification process demonstrated that all critical parameters remain well within the specified limits.

The finalized design dimensions are as follows:
\begin{itemize}
    \item \textbf{Cladding Thickness:} $90 \mu m$.
    \item \textbf{Plenum Height:} $1 m$.
\end{itemize}

With the selected dimensions, we extended the computation to simulate an uptime of 2 years instead of the initial 1-year design. The results demonstrate the performance and safety of the fuel pin under prolonged operational conditions:

\begin{itemize}
    \item \textbf{Maximum Fuel Temperature:} 2553.260 K (increased from 2480.096 K).
    \item \textbf{Maximum Cladding Temperature:} 912.696 K (unchanged).
    \item \textbf{Plenum Pressure:} 5.409 MPa (limit exceeded, increased from 2.792 MPa).
    \item \textbf{Maximum Instantaneous Cladding Plastic Strain:} 0.000\% (unchanged).
    \item \textbf{Maximum Cladding Volumetric Swelling:} 20.118\% (limit exceeded, increased significantly from 2.900\%).
    \item \textbf{Maximum Coolant Velocity:} 5.558 m/s (increased from 4.779 m/s).
    \item \textbf{Minimum Gap Thickness:} 367.927 microns (decreased from 386.495 microns).
    \item \textbf{Burnup:} 128.268 GWd/tHM (increased from 64.134 GWd/tHM).
    \item \textbf{Fuel Yielding Due to Swelling:} 8.979\% (increased from 4.489\%).
    \item \textbf{Time to Rupture:} 4.23 years (decreased significantly from 51.98 years).
\end{itemize}

\textbf{Comparison and Observations:}
\begin{itemize}
    \item The maximum fuel temperature increased slightly, indicating a higher thermal load on the fuel, likely due to increased burnup.
    \item The plenum pressure and cladding volumetric swelling both exceeded design limits, indicating that the fuel pin design is insufficient for extended irradiation cycles without adjustments.
    \item The burnup doubled, as expected
    \item The time to rupture decreased drastically from 51.98 years to 4.23 years, showing that thermal creep becomes a significant concern under these conditions.
    \item While the coolant velocity increased, it remained within acceptable operational ranges, highlighting that the thermal-hydraulic design is not a limiting factor in this scenario.
    \item The decreased gap thickness indicates increased mechanical interaction between the fuel and cladding, which can exacerbate the risks associated with swelling and fuel-cladding mechanical contact.
\end{itemize}

In summary, the significant rise in cladding swelling, plenum pressure, and reduced time to rupture indicate that this design requires further optimization to handle extended operation cycles safely.