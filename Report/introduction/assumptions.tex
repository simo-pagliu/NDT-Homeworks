\subsection{Assumptions}
The following assumptions were adopted to address the given problem:

\subsubsection*{Steady-State Solution for Gas in Grains}
\begin{itemize}
    \item[$\hookrightarrow$] The rate equation for the gas within the grains (PDE) was analyzed under steady-state conditions. This approach was chosen because the objective is to size the plenum for a 1-year operation without focusing on the temporal evolution of the function.
\end{itemize}

\subsubsection*{Fission Rate Calculation}
\begin{itemize}
    \item[$\hookrightarrow$] The fission rate was determined using the formula:
    \begin{equation*}
        \text{Fission Rate} = \Sigma_f \Phi_{\text{avg}}  
    \end{equation*}
    \item[$\hookrightarrow$] The macroscopic fission cross section was derived using data from the JANIS database.
    \item[$\hookrightarrow$] The average flux was calculated by assuming a uniform power and flux profile.
\end{itemize}

\subsubsection*{Material Properties and Geometry}
\begin{itemize}
    \item[$\hookrightarrow$] Material properties were treated as temperature-dependent and modeled using the provided empirical correlations.
    \item[$\hookrightarrow$] Axial power and neutron flux profiles were assumed to remain constant over time.
    \item[$\hookrightarrow$] Initial helium pressure and temperature in the fuel-cladding gap were considered as specified.
    \item[$\hookrightarrow$] Simplifications in geometry were introduced, such as neglecting axial deformation, to facilitate computation.
\end{itemize}
