\subsection{Assumptions}
To address the given problem, the following assumptions were made:

\textbf{Steady State Solution for Gas in Grains:}

\begin{itemize}
\item[$\hookrightarrow$] The rate equation for the gas remaining in the grains (PDE) was solved in steady state conditions, given that we just want to size the plenum for 1-year operation and are not interested in the specific behavior in time of the function.
\end{itemize}

\textbf{Fission Rate Calculation:}

\begin{itemize}
\item[$\hookrightarrow$] The fission rate was calculated by using the formula (macroscopic cross section * average flux).
\item[$\hookrightarrow$] The macroscopic cross section of fission was computed from data taken from the JANIS database.
\item[$\hookrightarrow$] The average flux was evaluated considering power and flux profile to be equal.
\end{itemize}

\textbf{Material Properties and Geometry:}

\begin{itemize}
\item[$\hookrightarrow$] Material properties are temperature-dependent and were modeled using provided empirical correlations.
\item[$\hookrightarrow$] Axial power and neutron flux profiles remain constant over time.
\item[$\hookrightarrow$] Initial helium pressure and temperature in the fuel-cladding gap were assumed as specified.
\item[$\hookrightarrow$] Simplifications in geometry, such as neglecting axial deformation, were made to ease computation.
\end{itemize}
